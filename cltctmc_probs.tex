\documentclass{article}
\usepackage{amsmath, fullpage, amssymb}
%\DeclareMathOperator{Pr}{\text{Pr}}
\begin{document}
\section*{Labeled transition PMF}
Suppose $\{X_t\}$ is a two-state
continuous time Markov chain with infinitesimal generator
\begin{equation*}
\Lambda = \begin{pmatrix}
 -\lambda_1 & \lambda_1 \\
 \lambda_2 & -\lambda_2
\end{pmatrix}
\end{equation*}
with $\lambda_1 > 0$ and $\lambda_2 > 0$. We take $X_t$ to have initial
state $X_o = 1$. Let $\{N_t\}$ be a process
counting the number of transitions $\{(1,2),(2,1)\}$ made in $\{X_t\}$.
We derive an alternate expression for the probability mass function
of $N_t$, which we denote as $q_n(t)$ = $Pr(N_t = n | X_t = 1)$. Recall
the Laplace transform of $q_n$:
\begin{equation*}
f_{2k}(s) = \frac{\lambda_1^k \lambda_2^k}{(s + \lambda_1)^{k+1}(s + \lambda_2)^k} \quad \quad
f_{2k-1}(s) = \frac{\lambda_1^k \lambda_2^{k-1}}{(s + \lambda_1)^k(s + \lambda_2)^k}
\end{equation*}
Using the convolution property of the Laplace transform, we express
\begin{flalign*}
q_{2k-1}(t) =& \frac{\lambda_1^k\lambda_2^{k-1}}{(k-1)!(k-1)!)}
 \int_0^t \tau^{k-1} e^{-\lambda_1 \tau} (t - \tau)^{k-1} e^{-\lambda_2(t - \tau)}d\tau \\
=& \frac{\lambda_1^k\lambda_2^{k-1}}{\Gamma(k)^2}e^{-\lambda_2 t}
 t^{2k - 1}\int_0^1 s^{k-1}(1 - s)^{k-1}e^{t(\lambda_2 - \lambda_1)s}ds \\
=& \frac{\lambda_1^k\lambda_2^{k-1}}{\Gamma(2k)}e^{-\lambda_2 t}
 t^{2k - 1} E[e^{t(\lambda_2 - \lambda_1)X}], \quad X \sim Beta(k, k) \\
=& \frac{\lambda_1^k\lambda_2^{k-1}}{\Gamma(2k)}e^{-\lambda_2 t}
 t^{2k - 1}\, _1F_1\left(k, 2k, t(\lambda_2 - \lambda_1)\right).
\end{flalign*}
A similar calculation shows
\begin{equation*}
q_{2k}(t) = \frac{\lambda_1^k\lambda_2^k}{\Gamma(2k + 1)}e^{-\lambda_2t}
t^{2k}\, _1F_1\left(k + 1, 2k + 1, t(\lambda_2 - \lambda_1)\right)
\end{equation*}
The function $_1F_1(a, b, x)$ is the confluent hypergeometric function.
\subsection*{Evaluating $_1F_1(k, 2k, x)$}

In certain cases the confluent hypergeometric function shares a connection 
with the spherical Bessel function of the first kind, commonly denoted 
as $I_k(x)$. In particular,
\begin{equation*}
  _1F_1(k, 2k, x) = e^{\frac{x}{2}}\left(\frac{1}{4}x\right)^{\frac{1}{2} - k}
\Gamma\left(k + \frac{1}{2}\right)I_{k-\frac{1}{2}}\left(\frac{1}{2}x\right).
\end{equation*}
The Bessel function $I_\alpha$ is not necessarily cheap to compute. Fortunately 
this Bessel function obeys a linear difference equation. For $n = 1, 2, \ldots$,
\begin{equation*}
\frac{2\nu}{x}I_\nu(x) + I_{\nu + 1}(x) = I_{\nu - 1}(x)
\end{equation*}
This recursion is stable in the backward direction when $x > 0$. The forward direction 
suffers from heavy cancellation when $x > 0$ and must be avoided.
\subsection*{Evaluating $_1F_1(k + 1, 2k + 1, x)$}
The confluent hypergeometric function obeys a variety of contiguous relations. Given values
$_1F_1(k, 2k, x)$ and $_1F_1(k + 1, 2k + 2, x)$, we may compute
\begin{equation*}
  _1F_1(k + 1, 2k + 1, x) = \,_1F_1(k, 2k, x) + \frac{x}{4k + 2}\,_1F_1(k + 1, 2k + 2, x)
\end{equation*}

\end{document}
